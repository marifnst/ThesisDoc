%
% Halaman Abstract
%
% @author  Andreas Febrian
% @version 1.00
%

\chapter*{ABSTRACT}

\vspace*{0.2cm}

\noindent \begin{tabular}{l l p{11.0cm}}
	Name&: & \penulis \\
	Program&: & \program \\
	Title&: & \judulInggris \\
\end{tabular} \\ 

\vspace*{0.5cm}

\noindent Batik is one of Indonesia traditional fabric which recognized by UNESCO as Masterpiece of Oral and Intangible Heritage of Humanity. Batik has many textures such as kawung, parangkusumo, truntum, tambal, pamiluto, parang liris and udan nitik. Moreover, every province in indonesia has their own texture that represent their culture. The problem of batik classification is appeared caused by variation in texture or province and no useful information and knowledge can support batik recognition. So, CNN is proposed to be used in learning texture for batik recognition. CNN will process input into kernel and processed into common layer called convolution and subsampling and the output layer can be used for batik recognition. For CNN computation, deepleraning4j will be used so learning result can be stored into binary file and can be used to evaluate batik texture for batik recognition.\\

\vspace*{0.2cm}

\noindent Keywords: \\ 
\noindent Android, Deep Learning, Convolution Neural Network
\newpage