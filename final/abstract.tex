%
% Halaman Abstract
%
% @author  Andreas Febrian
% @version 1.00
%

\chapter*{ABSTRACT}

\vspace*{0.2cm}

\noindent \begin{tabular}{l l p{11.0cm}}
	Name&: & \penulis \\
	Program&: & \program \\
	Title&: & \judulInggris \\
\end{tabular} \\ 

\vspace*{0.5cm}

\noindent Batik is one of Indonesia traditional fabric which recognized by UNESCO as Masterpiece of Oral and Intangible Heritage of Humanity. Batik has many textures such as kawung, parangkusumo, truntum, tambal, pamiluto, parang liris and udan nitik. Moreover, every province in indonesia has their own texture that represent their culture. The problem of batik classification is appeared caused by variation in texture or province and no useful information, knowledge or technology that can support batik recognition. Deep Learning is on of machine learning area which has significant improvement today and able to modeling complex data in real world. Deep Learning has been implemented in many research such as face recognition, action recognition, gesture recognition, or integrated with metaheuristic algorithm. Unfortunately, deep learning research for mobile technology is still few adopted. So, Fast R-CNN is proposed to be used in learning texture for batik recognition with better accuracy and less computation time. Fast R-CNN will be implemented into mobile technology so batik information can be reached easily by Indonesia people.\\

\vspace*{0.2cm}

\noindent Keywords: \\ 
\noindent Android, Deep Learning, Convolution Neural Network
\newpage