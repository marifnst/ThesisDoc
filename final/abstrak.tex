%
% Halaman Abstrak
%
% @author  Andreas Febrian
% @version 1.00
%

\chapter*{Abstrak}

\vspace*{0.2cm}

\noindent \begin{tabular}{l l p{10cm}}
	Nama&: & \penulis \\
	Program Studi&: & \program \\
	Judul&: & \judul \\
\end{tabular} \\ 

\vspace*{0.5cm}

\noindent Kain batik merupakan salah satu warisan kebudayaan Indonesia yang menjadi salah satu warisan dunia menurut UNESCO. Batik memiliki beberapa motif seperti motif kawung, motif parangkusumo, motif truntum, motif tambal, motif pamiluto, motif parang, motif liris maupun motif udan nitik. Selain banyak variasi motif batik, daerah asal batik juga beragam dan memiliki makna yang memiliki keterikatan dengan daerah asal batik tersebut. Dengan beragamnya daaerah motif maupun daerah asal batik, akan menjadi sulit untuk mengetahui daerah asalah batik tersebut jika tidak didukungnya informasi maupun pengetahuan terkait batik. Dengan menggunakan metode convolution neural network, diharapkan dapat membantu mempelajari variasi motif batik untuk mendeteksi daerah asal batik tersebut. Convolution neural network akan melakukan pembelajaran image dengan memecah gambar menjadi lebih kecil atau kernel dan diproses kedalam 2 layer utama (layer konvolusi dan subsamplling) hingga mencapai output layer dan dilakukan evaluasi terhadap data tes. Untuk melakukan komputasi CNN, digunakan library deeplearning4j sehingga hasil pembelajaran CNN bisa disimpan dalam file biner yang bisa digunakan kembali untuk melakukan evaluasi motif batik.   \\

\vspace*{0.2cm}

\noindent Kata Kunci: \\ 
\noindent Android, Deep Learning, Convolution Neural Network \\

\newpage