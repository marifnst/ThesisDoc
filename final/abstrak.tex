%
% Halaman Abstrak
%
% @author  Andreas Febrian
% @version 1.00
%

\chapter*{Abstrak}

\vspace*{0.2cm}

\noindent \begin{tabular}{l l p{10cm}}
	Nama&: & \penulis \\
	Program Studi&: & \program \\
	Judul&: & \judul \\
\end{tabular} \\ 

\vspace*{0.5cm}

\noindent Kain batik merupakan salah satu warisan kebudayaan Indonesia yang menjadi salah satu warisan dunia menurut UNESCO. Batik memiliki beberapa motif seperti motif kawung, motif parangkusumo, motif truntum, motif tambal, motif pamiluto, motif parang, motif liris maupun motif udan nitik. Selain banyak variasi motif batik, daerah asal batik juga beragam dan memiliki makna yang memiliki keterikatan dengan daerah asal batik tersebut. Dengan beragamnya daaerah motif maupun daerah asal batik, akan menjadi sulit untuk mengetahui jenis motif maupun daerah asal batik tersebut jika tidak didukungnya informasi, pengetahuan maupun teknologi yang mendukung pengetahuan tentang batik. Deep learning adalah salah satu area pembelajaran mesin yang sedang berkembang pesat dan memiliki kemampuan untuk melakukan pemodelan data kompleks dari dunia nyata. Area penelitian deep learning banyak mempengaruhi banyak aspek seperti pengenalan wajah, aksi, gestur maupun integrasi deep learning dengan metode heuristik. Sayangnya, meskipun penelitian terkait deep learning tersebut akan sangat penting dan bermanfaat ketika diimplementasikan pada perangkat mobile, masih sangat sedikit penelitian yang mengadopsi teknik deep learning pada perangkat mobile. Oleh karena itu, dengan menggunakan Fast R-CNN yang merupakan pengembangan dari Convolutional Neural Network, diharapkan deteksi motif batik akan memberikan akurasi dengan waktu komputasi yang lebih cepat. Selain itu, implementasi Fast R-CNN akan dilakukan pada perangkat mobile dengan tujuan informasi batik lebih mudah dijangkau oleh masyarakat luas.  \\

\vspace*{0.2cm}

\noindent Kata Kunci: \\ 
\noindent Android, Deep Learning, Convolution Neural Network \\

\newpage