%-----------------------------------------------------------------------------%
\chapter{\babSatu}
%-----------------------------------------------------------------------------%

%-----------------------------------------------------------------------------%
\section{Latar Belakang}
%-----------------------------------------------------------------------------%
Berdasarkan \cite{deepx}, pada saat ini model statistik yang memberikan akurasi terbaik untuk melakukan pengenalan kebiasaan manusia maupun objek umum dibangun menggunakan metode deep learning, yaitu salah satu area pembelajaran mesin yang sedang berkembang pesat dan  memiliki kemampuan untuk melakukan pemodelan data kompleks dari dunia nyata. Area penelitian yang dipengaruhi oleh deep learning mencakup pengenalan wajah \cite{face_occlusion}, aksi \cite{har_mocap_fnn}, gestur \cite{hand_gesture} maupun integrasi deep learning dengan metode heuristik \cite{meta_cnn}. Sayangnya, meskipun penelitian terkait deep learning tersebut akan sangat penting dan bermanfaat ketika diimplementasikan pada perangkat mobile, masih sangat sedikit penelitian yang mengadopsi teknik deep learning pada perangkat mobile.

Pendekatan deep learning pada teknologi mobile pada saat ini banyak memiliki kekurangan \cite{deepx}. Salah satu solusinya adalah melakukan proses deep learning pada cloud, tetapi teknologi cloud sangat tidak praktis untuk jangka panjang menyangkut biaya pelayanan cloud dan konsumsi bandwidth internet. Selain itu, ketika terjadi permasalahan pada jaringan dan server cloud tidak aktif, maka aplikasi pada perangkat mobile akan menjadi tidak berguna. Untuk melakukan komputasi deep learning pada perangkat CPU lokal memang memungkinkan untuk beberapa skenario, tetapi dibutuhkan usaha dan kemampuan yang mumpuni, selain itu penggunaan model tersebut sangat terbatas dan tidak bisa digunakan secara umum oleh model deep learning lainnya. Padahal, permasalahan kompleks yang diselesaikan model deep learning adalah hal yang biasanya dibutuhkan pada perangkat mobile.

Batik merupakan kain bergambar yang memiliki gaya, warna dan tekstur dimana proses pembuatannya dilakukan secara manual maupun menggunakan mesin dan merupakan salah satu kain tradisional yang dimiliki oleh negara indonesia. Batik sudah ditetapkan sebagai Warisan Kemanusiaan untuk Budaya Lisan dan Nonbendawi oleh UNESCO sejak 2 Oktober 2009. Dengan keberagaman suku dan budaya di Indonesia, menyebabkan Indonesia memiliki variasi motif batik yang sangat beraneka ragam dan memiliki makna simbolis berdasarkan daerah asal batik tersebut. Batik memiliki beberapa motif seperti motif kawung, motif parangkusumo, motif truntum, motif tambal, motif pamiluto, motif parang, motif liris maupun motif udan nitik yang menjadi contoh begitu beragamnya variasi motif batik yang ada di Indonesia.

Kontribusi utama dalam penelitian ini adalah mengusulkan metode pengenalan motif batik menggunakan Fast R-CNN yang merupakan pengembangan dari Convolutional Neural Network yang digabungkan dengan metode Spatial pyramid pooling networks (SPPnets). Fast R-CNN merupakan perbaikan dari R-CNN yang berfokus pada perbaikan waktu komputasi. Diharapkan dengan menggunakan Fast R-CNN untuk pengenalan motif batik, akurasi yang dihasilkan akan lebih baik ditambah waktu komputasi yang lebih cepat dibandingkan R-CNN. Selain itu, kontribusi lainnya adalah penggunaan metode deep learning pada perangkat mobile SoC, dimana Fast R-CNN akan digabungkan dengan metode DAD (Deep Architecture Decomposition) dan RLC (Runtime Layer Compression) untuk menyesuaikan komputasi deep learning dengan sumber daya yang terbatas pada perangkat mobile.

%-----------------------------------------------------------------------------%
\section{Rumusan Masalah}
%-----------------------------------------------------------------------------%
Pendahuluan pada butir 1.1 menimbulkan permasalahan-permasalahan yang perlu diselesaikan. Penelitian ini dilakukan untuk menemukan jawaban dari permasalahan-permasalahan tersebut, antara lain sebagai berikut:
\begin{enumerate}
	\item Bagaimana merancang algoritma Fast R-CNN untuk melakukan proses learning dan deteksi motif batik.
	\item Bagaimana integrasi algoritma Fast R-CNN dengan pendekatan Deep Architecture Decomposition \& RLS Runtime Layer Compression pada perangkat mobile
\end{enumerate}

%-----------------------------------------------------------------------------%
\section{Tujuan Penelitian}
%-----------------------------------------------------------------------------%
Berdasarkan permasalahan pada butir 1.2, tujuan-tujuan yang akan dicapai dalam penelitian ini adalah sebagai berikut.
\begin{enumerate}
	\item Implementasi Fast R-CNN untuk pengenalan motif batik
	\item Implementasi Fast R-CNN dengan integrasi pendekatan DAD \& RLC pada perangkat mobile untuk pengenalan motif batik
\end{enumerate}

%-----------------------------------------------------------------------------%
\section{Batasan Penelitian}
%-----------------------------------------------------------------------------%
Dalam melakukan penelitian ini, terdapat beberapa batasan-batasan yang digunakan. Batasan-batasan tersebut adalah sebagai berikut.
\begin{enumerate}
	\item Penelitian ini menggunakan data gambar dengan ukuran 300x300.
	\item Penelitian ini menggunakan metode Fast R-CNN untuk membangun model pengenalan motif batik.
	\item Penelitian ini hanya mendeteksi 5 motif batik, Ceplok, Kawung, Lereng, Nitik dan Parang.
	\item Penelitian ini menggunakan library faster-rcnn (https://github.com/rbgirshick/py-faster-rcnn) untuk proses Fast R-CNN dan NVIDIA cuDNN untuk implementasi deep learning pada perangkat mobile.
	\item Penelitian ini menggunakan perangkat NVIDIA Tegra K-1.
\end{enumerate}

%-----------------------------------------------------------------------------%
\section{Manfaat Penelitian}
%-----------------------------------------------------------------------------%
Manfaat dari penelitian ini adlah sebagai berikut:
\begin{enumerate}
	\item Dari segi keilmuan, penelitian ini akan membantu penggunaan model Fast R-CNN untuk pengenalan objek gambar selain batik. Selain itu, penggunaan Fast R-CNN dengan integrasi metode DAD dan RLC pada perangkat mobile dapat membantu kontribusi penelitian deep learning pada perangkat mobile.
	\item Dari sisi sosial, penelitian ini dapat membantu memberikan informasi pengenalan motif batik secara umum pada masyarakat menggunakan smartphone.
\end{enumerate}

%-----------------------------------------------------------------------------%
\section{Metodologi Penelitian}
%-----------------------------------------------------------------------------%
Untuk menjawab masalah yang terdapat pada rumusan masalah dan mencapai tujuan penelitian, penelitian ini dilakukan dengan metode eksperimen dengan langkah-langkah sebagai berikut:
\begin{enumerate}
	\item Membangun aplikasi untuk melakukan deteksi motif batik dengan Fast R-CNN
	\item Membangun Fast R-CNN dengan integrasi metode DAD dan RLC untuk mendeteksi motif batik pada perangkat mobileac
\end{enumerate}

%-----------------------------------------------------------------------------%
\section{Sistematika Penulisan}
%-----------------------------------------------------------------------------%
Sistematika penulisan laporan adalah sebagai berikut:
\begin{itemize}
	\item Bab 1 \babSatu \\
	Bab 1 berisi pendahuluan yang memberi penjelasan mengenai latar belakang penelitian ini dilakukan, masalah-masalah yang akan diselesaikan melalui penelitian ini, tujuan-tujuan penelitian yang akan dicapai, batasan penelitian, metode yang akan digunakan dalam penelitian ini, dan struktur penulisan proposal penelitian ini.
	\item Bab 2 \babDua \\
	Bab 2 berisi teori-teori yang berkaitan dalam penelitian yang akan dilaksanakan. Teori-teori tersebut antara lain Convolutional Neural Network, Fast Region-CNN, Thin Client dan teknologi perangkat mobile.
	\item Bab 3 \babTiga \\
	Bab 3 berisi usulan rancangan penelitian yang akan dilaksanakan. Rancangan tersebut terdiri dari langkah-langkah yang akan dilakukan, penjelasan teknis mengenai metode-metode yang akan diterapkan, dan tempat serta waktu penelitian.
%	\item Bab 4 \babEmpat \\
%	\item Bab 5 \babLima \\
%	\item Bab 6 \babEnam \\
%	\item Bab 7 \kesimpulan \\
\end{itemize}

%\todo{Tambahkan penjelasan singkat mengenai isi masing-masing bab.}

